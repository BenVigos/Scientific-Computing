%%%%%%%%%%%%%%%%%%%%%%%%%%%%%%%%%%%%%%%%%%%%%%%%%%%%%%
%       UNIVERSITY OF AMSTERDAM ASSIGNMENT TEMPLATE      %
%            (VERSION WITH LOGO INTEGRATION)             %
%%%%%%%%%%%%%%%%%%%%%%%%%%%%%%%%%%%%%%%%%%%%%%%%%%%%%%

% NOTE: You must have an image file named "uva_logo.png"
% in the same directory as this .tex file for it to compile.

\documentclass[11pt, a4paper]{article}
\setlength{\headheight}{26.84232pt}
%=========== REQUIRED PACKAGES ===========%
\usepackage[utf8]{inputenc} % For text encoding
\usepackage{amsmath}        % For advanced math typography
\usepackage{graphicx}       % To include images
\usepackage[margin=1in]{geometry} % To set page margins
\usepackage{fancyhdr}       % To customize headers and footers
\usepackage{lastpage}       % To get the total number of pages
\usepackage{hyperref}       % For clickable links and table of contents
\hypersetup{
    colorlinks=true,
    linkcolor=blue,
    filecolor=magenta,      
    urlcolor=cyan,
}

%=========== TITLE PAGE INFORMATION ===========%
%--- Fill in your details here ---
\newcommand{\studentName}{Name} % [cite: 1]
\newcommand{\studentLastname}{Lastname} % [cite: 2]
\newcommand{\studentID}{12345678} % [cite: 2]
\newcommand{\reportTitle}{Your Report Title Here} % [cite: 5]
\newcommand{\courseName}{Introduction to Computational Science} % [cite: 10, 11]
\newcommand{\courseCode}{XXXX} % [cite: 13]
\newcommand{\lecturerName}{Valeria Krzhizhanovskaya} % [cite: 8]


%=========== HEADER AND FOOTER CONFIGURATION ===========%
\pagestyle{fancy} % Apply the fancy page style
\fancyhf{} % Clear all header and footer fields

%--- Header ---
% Logo in the left header [cite: 16, 27]
\lhead{\includegraphics[height=0.8cm]{Hippo_JPG-uvalogo_regular_p_en.jpg}} 
\rhead{ASSIGNMENT | REPORT} % [cite: 24, 28]

%--- Footer ---
\lfoot{\studentName\ \studentLastname} % [cite: 25, 29]
\rfoot{PAGE \thepage\ OF \pageref{LastPage}} % [cite: 14, 26, 30]

%--- Rule settings ---
\renewcommand{\headrulewidth}{0.4pt} % Line under the header
\renewcommand{\footrulewidth}{0.4pt} % Line above the footer


%=========== DOCUMENT START ===========%
\begin{document}

%=========== TITLE PAGE ===========%
\begin{titlepage}
    \centering
    \vspace*{2cm}
    
    % University Logo [cite: 3]
    \includegraphics[width=7cm]{Hippo_JPG-uvalogo_regular_p_en.jpg}\par
    
    \vspace{2.5cm}
    
    % Report Type [cite: 4]
    {\Huge\bfseries REPORT\par}
    
    \vfill % Pushes content to the vertical center
    
    % Report Title [cite: 5]
    {\LARGE\bfseries \reportTitle\par}
    
    \vfill
    
    % Author and Course Details
    \begin{minipage}{0.6\textwidth}
        \begin{flushleft} \large
            \emph{Student:} \\ % [cite: 1]
            \studentName\ \studentLastname \\ % [cite: 1, 2]
            Student ID: \studentID \\[1cm] % [cite: 2]
            
            \emph{Lecturer:} \lecturerName \\[1cm] % [cite: 7, 8]
            
            \emph{Course:} \\ % [cite: 9]
            \courseName \\ % [cite: 10, 11]
            
        \end{flushleft}
    \end{minipage}
    
    \vfill
    
    % Date [cite: 6]
    {\large \today}
    
    % Set the footer for the title page specifically
    \thispagestyle{fancy}
\end{titlepage}


%=========== TABLE OF CONTENTS ===========%
\newpage
\tableofcontents
\newpage

%=========== MAIN CONTENT SECTIONS ===========%
% The structure below is based on the second page of the document.

\section{Introduction} % [cite: 17]
\label{sec:introduction}
Your introduction goes here. State the problem, your approach, and the structure of the report.

\section{Theory} % [cite: 18]
\label{sec:theory}
Explain the theoretical concepts, models, and equations relevant to your work. For example, the formula for gravitational force is $F = G \frac{m_1 m_2}{r^2}$.

\section{Numerical methods} % [cite: 19]
\label{sec:methods}
Describe the numerical methods and algorithms you used to solve the problem. Include implementation details.

\section{Results and discussion} % [cite: 20]
\label{sec:results}
Present your findings using figures, tables, and graphs. Discuss what these results mean and interpret them in the context of the theory.

\section{Conclusions} % [cite: 22]
\label{sec:conclusions}
Summarize your key findings and conclude the report. You can also suggest potential future work.

%=========== REFERENCES ===========%
% Using the `thebibliography` environment for manual references.
\begin{thebibliography}{9} % [cite: 23]

\bibitem{key1}
Author, A. N. (Year). \textit{Title of work}. Publisher.

\bibitem{key2}
Second, A. U. Thor, \& Third, C. O. Author (Year). Title of article. \textit{Journal Name}, Volume(Issue), pages.

\end{thebibliography}

\end{document}